\section{Related Work}
\subsection{Query Optimizations}
There has been extensive work in optimizing relational queries since the early '70s~\cite{Chaudhuri:1998}. A large class of optimizations include exploiting commutativity among operators~\cite{Chaudhuri:1994,Yan:1995},reducing multi-block queries to single block~\cite{Kim:1982,Muralikrishna:1992}, using semijoin techniques to optimize multi-block queries~\cite{Mumick:1994,Seshadri:1996}, query materialization~\cite{Chaudhuri:1995,Phan:2008}, query indexing~\cite{SELLIS1988175,Bertino:1989,Fang:2008}. Several approaches address optimizations of stored procedures (also called user-defined predicates) in relational systems~\cite{Hellerstein:1993,Chimenti:1989}. While these aforementioned techniques are powerful in optimizing relational queries, traditional query optimizers treat non-relational code as a black-box.

\subsection{MapReduce Optimizations}
MapReduce is a relatively young framework, but it quickly became very popular programming model for big data analytics~\cite{Gates:2009}. Optimizing MapReduce is of a great importance and various approaches have been proposed to improve the performance of MapReduce jobs. They include techniques to improve the scheduling of task execution~\cite{Isard:2009,Zaharia:2008} and to efficiently perform joins and indexing~\cite{Dittrich:2010,Floratou:2011}. The recent work~\cite{Jahani:2011} suggests that traditional query optimizations are not applicable in MapReduce jobs because of the user written operators. It further proposes several optimizations of \emph{map()} function by targeting data-centric programming idioms~\cite{Jahani:2011}. Our work shows that time spent in non-relational code takes a large fraction of data center time and by considering optimizations opportunities across relational and non-relational parts can significantly improve the performance of MapReduce jobs.
