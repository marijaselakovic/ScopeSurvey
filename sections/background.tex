\section{SCOPE \label{sec:Scope}}

SCOPE~\cite{SCOPE} is used internally within Microsoft and is now transitioning into an external offering as U-SQL~\cite{usql}.
Its relational part is very similar to SQL, enough so that we will ignore any differences.
The non-relational part is C\#~\cite{Hejlsberg:2010:CPL:1951915}: all expressions in a script are written as C\# expressions.
In addition, SCOPE allows user-defined functions, {\em UDFs}, and user-defined operators, {\em UDOs}.

\subsection{Execution of a Script}
A script is implemented as a directed acyclic graph (DAG) where each vertex is a set of operators implemented on the same physical (or virtual) machine. We use the term {\it node} for the physical or virtual machine that a vertex is implemented on.
The edges of the DAG are communication channels that use a high-speed communication network between nodes.
The operators within a vertex are the end product of a very sophisticated optimizer \cite{}; expressions written within a certain construct in the script may end up being executed in vertices that do not correspond to the construct in a simple manner.
For instance, a sub-expression from a {\tt WHERE} clause, {\em filter}, may be {\it promoted} into a vertex which extracts an input table from a data source, whereas the rest of the filter may be in a vertex that is many edges distant from the input layer.
An execution of a script is called a {\em job}.

\subsection{C++ vs. C\#}
\label{sec:nativenonnative}
Much like Hadoop streaming~\cite{hadoop_stream}, SCOPE jobs consist of multiple runtimes and languages and while the details of this paper are about SCOPE, the general problem is shared among many big data systems.
The SCOPE compiler attempts to generate both C++ and C\# operators for the same source-level construct.
Each operator, however, must execute either entirely in C\# or C++: mixed code is not provided for.
Thus, when possible, the C++ operator is preferred because the data layout in stored data uses C++ data structures.
Thus, for example, a simple projection of a subset of the columns can be done entirely
without using the CLR.
But when a script contains a C\# expression that cannot be converted to a C++ function, such as in Figure~\ref{fig:non-native}, the CLR must be started, each row in the input table must be converted to a C\# representation, i.e., a C\# object representing the row must be created, and then the C\# expression can be evaluated in the CLR.

Because this can be inefficient, the SCOPE runtime contains C++ functions that are semantically equivalent to a subset of the .NET Framework methods that are frequently used; these are called {\it intrinsics}.
The SCOPE compiler then emit calls to the (C++) intrinsics in the C++ generated operator, which is then used at runtime in preference to the C\# generated operator.
(As opposed to using {\it interop} to execute native code from within the CLR.)

\subsection{Compiler/Optimizer Communication}
In general, the C\# code is compiled as a black box: no analysis/optimization is peformed at this level.
One consequence is that any calls to a
UDF within a SCOPE expression (filter predicate, projection function) require the operator containing
the call to be implemented in C\#.

