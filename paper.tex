\documentclass[sigconf, anonymous]{acmart}

\usepackage{booktabs} % For formal tables

\usepackage{arev}
% Copyright
%\setcopyright{none}
%\setcopyright{acmcopyright}
%\setcopyright{acmlicensed}
\setcopyright{rightsretained}
%\setcopyright{usgov}
%\setcopyright{usgovmixed}
%\setcopyright{cagov}
%\setcopyright{cagovmixed}



% DOI
%\acmDOI{10.475/123_4}

% ISBN
%\acmISBN{123-4567-24-567/08/06}

%Conference
%\acmConference[WOODSTOCK'97]{ACM Woodstock conference}{July 1997}{El
%  Paso, Texas USA} 
%\acmYear{1997}
%\copyrightyear{2016}
%
%\acmPrice{15.00}

%\title: Think Outside the Box: Exploring and Exploiting Cross-Language Optimizations in SCOPE



\begin{document}
\title{Exploring Cross-Language Optimizations in Big Data System: An Empirical Study of SCOPE}
%\titlenote{Produces the permission block, and
%  copyright information}
%\subtitlenote{The full version of the author's guide is available as
%  \texttt{acmart.pdf} document}


\author{Marija Selakovic}

%\orcid{1234-5678-9012}
\affiliation{%
  \institution{TU Darmstadt}
  \state{Germany} 
}
\email{m.selakovic89@gmail.com}

\author{Michael Barnett}

\affiliation{%
  \institution{Microsoft Research}
  \state{USA} 
}
\email{Michael.Barnett@microsoft.com}

\author{Madan Musuvathi}

\affiliation{%
  \institution{Microsoft Research}
    \state{USA} 
    }

  
\email{madanm@microsoft.com}



\author{Todd Mytkowicz}

\affiliation{%
  \institution{Microsoft Research}
    \state{USA} 
    }

  
\email{toddm@microsoft.com}


% The default list of authors is too long for headers}
\renewcommand{\shortauthors}{M. Selakovic et al.}


\begin{abstract}

To perform big data analytics jobs languages that combine relational (SQL) with non-relational code (Java, C$\sharp$, Scala) have been developed.
These languages become very popular among developers because they allow writing big data jobs in a declarative manner, while the decision on how to distribute the job over processors is completely taken by the runtime. Even though SQL optimizations have been extensively researched over past several decades, still the little is known on why and how to optimize these jobs across different languages. In this paper, we illustrate that job stages with non-relational code take between 45-70\% of data center CPU time by empirically studying over 3 millions of SCOPE jobs across five data centers in Microsoft. We further explore the potential for SCOPE optimization by generating more native code from the non-relational part. Finally, we present X case studies showing that triggering more generation of native code in these jobs yields to significant performance improvement that ranges between Y and Z\%.

\end{abstract}

%
% The code below should be generated by the tool at
% http://dl.acm.org/ccs.cfm
% Please copy and paste the code instead of the example below. 
%



%\keywords{ACM proceedings, \LaTeX, text tagging}


\maketitle

\section{Introduction \label{sec:intro}}
Large-scale data-processing frameworks, such as MapReduce~\cite{MapReduce}, SCOPE~\cite{SCOPE}, Hadoop~\cite{Dittrich:2010}, Spark~\cite{zaharia2010spark}, have become an integral part of computing today. One reason for their immense popularity is that they provide a simplified programming model that greatly simplifies the distribution and fault-tolerance of big-data processing. For instance, frameworks like SCOPE and Spark provide a SQL-like declarative interface for specifying the relational skeleton of data-processing jobs while providing extensibility by supporting expressions and functions written in general-purpose languages like C\#, Java, or Scala. 

%\emph{MapReduce} framework has become an immensely popular for easy development of scalable parallel applications that process large amounts of data. 
%The advantage of MapReduce is that it isolates the details of running a big data application as a distributed program. 
%To ease the use of MapReduce, several projects (Apache Pig, Apache Hive or SCOPE) provide high-level declarative interfaces on top of the MapReduce framework. 
%This means that big data processing jobs are composed of queries expressed in an SQL-like declarative language with expressions written in languages like Java, C\# or Scala.
The relational aspect is crucial: it is what enables the automatic parallelization for efficiently scaling out to arbitrary amounts of data.
Big data systems assume that the non-relational part is written carefully enough so that it does not violate the assumptions needed for automatic parallelization: e.g., programmers must write their non-relational logic to be deterministic and insensitive to the ordering of the input.

However, these systems are known to lag far behind traditional database systems in runtime efficiency~\cite{Jahani:2011,Pavlo:2009}, primarily because of the flexibility of the programming model they support. For instance,  a key bottleneck in Spark is neither the disk nor the network, but the time spent by the CPU on compression/decompression of data, serialization/deserialization of the input into/from Java objects, and the JVM garbage collection~\cite{ousterhout-nsdi15}. SCOPE,  described more fully in Section~\ref{sec:Scope}, supports a hybrid native (C++) and C\# runtime partly to alleviate this overhead.
Like SCOPE, Hadoop Streaming lets programmers write programs in a mix of languages\cite{hadoop_stream}. 
Our analysis shows that this cross-language interaction (in SCOPE, between the native and C\# runtimes) is a significant cost in the overall system.  
Equally importantly, the presence of non-relational code blocks the powerful relational optimizations implemented in these data-processing runtimes, e.g. ~\cite{guo2012spotting}. 


The goal of this work is to study and better understand the key performance bottlenecks in modern data-processing systems, and demonstrate the potential for cross-language optimizations. While this paper is primarily about SCOPE, we believe our results and optimizations generalize to other data-processing systems. 
%SCOPE (Structured Computations for Parallel Execution) is a query language that combines relational logic written in SQL with user expressions written in C\#.
SCOPE is the key data-processing system used at Microsoft running at least half a million jobs daily on serveral Microsoft data centers. 
Figure~\ref{fig:example} shows a simple example of a SCOPE program (hereafter referred to as a {\em script}) that interleaves relational logic with C\# expressions.

%To achieve this goal, we empiricaly analyse over 3,000,000 SCOPE jobs that run across five data centers at Microsoft during one week. 

%The language supports MapReduce programming model and is developed at Microsoft for big data processing. 
%Nowadays, it is used for almost 450,000 daily jobs running on several data centers in Microsoft.


%One source of inefficiency is due to non-relational code found in big data queries. 
%Despite powerfulness of SQL optimizations, such optimizations can not be applied to arbitrary code written in an imperative language.
%This calls for cross language optimizations in order to obtain a high-level efficiency of processing big data jobs.
%Moreover, we believe that increasing hardware and computing power is no longer a sustainable way of improving the efficiency of big data processing . 

%The goal of this work is to understand how the time is spent in big-data jobs and where is a potential for cross language optimizations.

%As described later in Section~\ref{sec:background}, the stage of SCOPE job can run as native (C++) if it does not contain any non-relational code. 
%However, this is unlikely the case and to run C\# code in SCOPE script the runtime performs data serialization from C++ format to format required by .NET runtime. 
%This process of data serialization and deserialization is inherently very expensive and to increase the number of job stages that run as native, the SCOPE compiler provides C++ implementation for a subset of .NET framework methods. 
%To illustrate when SCOPE script runs as native vs. non-native, consider examples in Figure~\ref{fig:example}. 
%Both code fragments are very simple SCOPE jobs. 
%The script in Figure~\ref{fig:native} runs as native, because SCOPE compiler has C++ implementation for \emph{String.isNullOrEmpty} method. On the other hand, the script in Figure~\ref{fig:non-native} has a call to \emph{Split} method, which is not optimized by the compiler and this causes entire script to run as non-native (C\# code).

\begin{figure}[ht]
 \begin{minipage}[b]{\linewidth}
  
   \begin{verbatim}
data = SELECT *
  FROM inputStream
  WHERE !String.IsNullOrEmpty(A) AND B == "Key1";
\end{verbatim}

    \subcaption{Predicate visible to optimizer.}
    \label{fig:native}
  \end{minipage}
  %
  \begin{minipage}[b]{\linewidth}
   \begin{verbatim}
data = SELECT *
  FROM inputStream
  WHERE M(A, B);

#CS
bool M(string x, string y) {
   return !String.IsNullOrEmpty(x) && y == "Key1";
}
#ENDCS
    \end{verbatim}

    \subcaption{Predicate invisible to optimizer.}
    \label{fig:non-native}
  \end{minipage}


\caption{Script Examples}
\label{fig:example}
\end{figure}

In Figure~\ref{fig:native}, the predicate in the {\tt WHERE} clause is subject to two potential optimizations:
\begin{enumerate}
\item The optimizer may choose to {\em promote} one (or both) of the conjuncts to an earlier part of the program, especially if either {\tt A} or {\tt B} are columns used for partitioning the data.
This can dramatically reduce the amount of data needed to be transferred from one vertex to another.
\item The SCOPE compiler has a set of methods that it considers to be {\em intrinsics}.
An intrinsic is a .NET method for which the SCOPE runtime has a semantically equivalent native function.
For instance, the method {\tt String.isNullOrEmpty} checks whether its argument is either null or else the empty string.
The corresponding native method is able to execute on the native data encoding which does not involve creating any .NET objects or instantiating the {\em CLR}, i.e., the .NET virtual machine.
\end{enumerate}

On the other hand, Figure~\ref{fig:non-native} shows a slight variation where the user implemented the predicate in a separate C\# method. Unfortunately, the SCOPE compiler treats the call to user-defined functions as a black box. As a result, both optimizations are disabled and the predicate is executed in a C\# virtual machine. The resulting serialization and data-copying costs can reduce the throughput of the job by as much as XXX percent. 


%has a call to \emph{Split} method, which the SCOPE compiler does not understand. This causes the ent


%which is not optimized by the compiler and this causes entire script to run as non-native (C\# code).

When facing a performance regression, a performance analyst first measures her system to understand where the bottlenecks are.  Then, she can act on those data to make her system faster.  While such an approach is well understood and easy to execute for desktop software, it becomes challenging in the context of a massive distributed system, like SCOPE.
This paper first describes \emph{how} we built a datacenter-wide profiler so we could even start to measure application bottlenecks.
Our new profiling infrastructure is a combination of offline static analysis of the executed code in addition to low-overhead online measurements captured by SCOPE's runtime.
Then, the paper describes the results of our profiling of over 3 million SCOPE programs across five data centers within Microsoft.
We find programs with non-relational code take between 45-70\% of data center CPU time.  

% %Despite knowing the cause of inefficiencies, still the little is known about how much time SCOPE jobs spend in non-relational code and what are the opportunities to trigger more generation of native code. 
% To answer these questions, we propose a new profiling infrastructure based on static analysis of job artifacts.
% By doing this, we are able to post-analyse large amount of jobs without introducing any additional overhead.
% Except the time spent in non-relational code, our profiling approach also surveys different sources of non-relational code in SCOPE jobs.
% It returns a list of most commonly used framework methods, for which having C++ implementation would trigger more generation of native code.
Finally, we discuss the effects of cross-language optimization based on \emph{method inlining}. 
By inlining a method call, the compiler/optimizer is now aware of the logic contained in the body of the method. 
We discuss the effectiveness of such optimizations in 5 case studies by optimizing jobs from 5 different teams at Microsoft. 

% Our experimental evaluation shows that non-native code takes between 45-75\% of data center CPU time. By further increasing the list of framework functions that have native implementation we can optimize up to Z\% of data center time.
% Finally, we illustrate that performance impact of inlining jobs to run as native is statistically significant and range between A and B\%. 

% As a consequence, our findings motivate future research in at least two ways.
% First, on tools and techniques that help developers write more efficient big data jobs by avoiding unnecessary generation of non-native code. 
% Second, on compiler optimizations that automatically generate native code from the non-relational logic.


% In summary, this work contributes the following:
% \begin{itemize}
% \item \emph{Profiling infrastructure for analyzing SCOPE jobs.} 
% We present the approach for profiling data centers based on static analysis of job artifacts. 
% The approach reports time spent in relational vs non-relational code and detects different sources of non-relational code for every jobs stage. 

% \item \emph{Reporting opporunities for cross-language optimizations.} We discuss two possible ways to enable further generation of native code. 
% First, we survey the most relevant framework functions relevant for native implementation. 
% Second, we present an analysis to find opportunities that trigger the generation of native code by \emph{inlining} user-written functions.  

% \item \emph{Empirical evidence.} 
% We illustrate by X case studies that enabling job stage to run as native code signifcantly improve the job performance by factors Y-K.
% \end{itemize}



\section{SCOPE \label{sec:Scope}}

SCOPE~\cite{SCOPE} is used internally within Microsoft and is now transitioning into an external offering as U-SQL~\cite{usql}.
Its relational part is very similar to SQL, enough so that we will ignore any differences.
The non-relational part is C\#~\cite{Hejlsberg:2010:CPL:1951915}: all expressions in a script are written as C\# expressions.
In addition, SCOPE allows user-defined functions, {\em UDFs}, and user-defined operators, {\em UDOs}.

\subsection{Execution of a Script}
A script is implemented as a directed acyclic graph (DAG) where each vertex is a set of operators implemented on the same physical (or virtual) machine. We use the term {\it node} for the physical or virtual machine that a vertex is implemented on.
The edges of the DAG are communication channels that use a high-speed communication network between nodes.
The operators within a vertex are the end product of a very sophisticated optimizer \cite{}; expressions written within a certain construct in the script may end up being executed in vertices that do not correspond to the construct in a simple manner.
For instance, a sub-expression from a {\tt WHERE} clause, {\em filter}, may be {\it promoted} into a vertex which extracts an input table from a data source, whereas the rest of the filter may be in a vertex that is many edges distant from the input layer.
An execution of a script is called a {\em job}.

\subsection{C++ vs. C\#}
Much like Hadoop streaming, SCOPE jobs consist of multiple runtimes and languages and while the details of this paper are about SCOPE, the general problem is shared among many big data systems.
The SCOPE compiler attempts to generate both C++ and C\# operators for the same source-level construct.
Each operator, however, must execute either entirely in C\# or C++: mixed code is not provided for.
Thus, when possible, the C++ operator is preferred because the data layout in stored data uses C++ data structures.
Thus, for example, a simple projection of a subset of the columns can be done entirely
without using the CLR.
But when a script contains a C\# expression that cannot be converted to a C++ function, such as in Figure~\ref{fig:non-native}, the CLR must be started, each row in the input table must be converted to a C\# representation, i.e., a C\# object representing the row must be created, and then the C\# expression can be evaluated in the CLR.

Because this can be inefficient, the SCOPE runtime contains C++ functions that are semantically equivalent to a subset of the .NET Framework methods that are frequently used; these are called {\it intrinsics}.
The SCOPE compiler then emit calls to the (C++) intrinsics in the C++ generated operator, which is then used at runtime in preference to the C\# generated operator.
(As opposed to using {\it interop} to execute native code from within the CLR.)

\subsection{Compiler/Optimizer Communication}
In general, the C\# code is compiled as a black box: no analysis/optimization is peformed at this level.
One consequence is that any calls to a
UDF within a SCOPE expression (filter predicate, projection function) require the operator containing
the call to be implemented in C\#.


\section{Profiling Infrastructure for Data Centers}
This section provides details on our profiling infrastructure for data centers that run big-data jobs. Because SCOPE compiler does not provide any instrumentation facility, we chose to collect profiling data by analyzing job artifacts. The benefit of doing this is at least twofold: we can derive data for a relatively large number of jobs since we do not require re-running them, and we can also answer more complex, but interesting questions, such as which job stages run as C++ vs. C\#.

\subsection{Scale of The Problem}
SCOPE jobs run on a distributed computing platform, called Cosmos, designed for storing and analyzing massive data sets. Cosmos run on five clusters consisting of thousands of commodity servers~\cite{SCOPE}. Important properties of Cosmos is scalability and performance. Nowadays, it stores exabytes of data across hundreds of thousands of physical machines. Cosmos runs millions of big-data jobs every week and almost half million jobs every day. It is used by more than 10,000 developers at Microsoft running very diverse workloads and scenarios.

Finding optimization opportunities that are applicable to such a large number of diverse jobs is a very challenging problem. To bring interesting conclusions we must ensure the scalability of our profiling infrastructure. To achieve this, the important aspect to consider is \emph{what type of information we should analyze}. In the following sections, we describe our major decisions when building infrastructure for profiling big data jobs.

\subsection{Job Artifacts}

After execution of a SCOPE job finishes, the runtime produces several artifacts that contain code and runtime information for every job stage. Job artifacts are indefinitely stored in Cosmos repository and this section gives an overview of each artifact we use to profile data center.
\paragraph{Job Algebra}

Job algebra is a graph representation of the job execution plan. Job vertices are presented as outer-most nodes in a graph. Each job vertex contains all operators that run inside that vertex and an operator can be either user-defined or native. Optionally, if all operators are native, the vertex can contain \emph{nativeOnly} flag indicating that entire vertex runs as native (C++). However, it does not distinguish between native and user-defined operators.


\paragraph{Runtime Statistics}
Runtime statistics file provides information on execution time for every job vertex and every operator inside the vertex. However, it includes only CPU times, which we use as a primary metric of a job execution time. 

%\paragraph{Binaries}
%The job repository stores a job and all third-party projects in binary format (dll) that contains both native and managed code (.NET) with the generated assembly code.


\paragraph{Generated C\# and C++ Code}
SCOPE runtime generates C\# and C++ code for every job. An artifact with C++ code has for every vertex a code region containing a C++ implementation of the vertex and another code region that provides class names for every operator that runs as C\#. An artifact with C\# code includes implementations of non-native operators and user-written classes and functions defined inside the script.

\subsection{Static Analysis}

\begin{figure}[ht]


\caption{High level picture of main steps during static analysis}
\label{fig:analysis}
\end{figure}


\subsubsection{Sources of Non-Native Code}

\begin{itemize}
\item .NET Framework Calls
\item User written functions
\item Custom processors, reducers, combiners, extractors,etc.
\end{itemize}

\subsubsection{Analysis of User-Written Code}
\section{Cross-Language Optimizations}
Cross-language optimizations can result in two different changes to the generated program that executes a query.

\subsection{Optimizations With Effects On Job Algebra}
The first changes the physical plan, i.e., the job algebra, because more information is
made available to the query optimizer.
For example, predicates might be pushed deeper into the DAG which can result in dramatic data reduction.


\subsection{Optimizations Without Effects On Job Algebra}
The second is that, for SCOPE, the set of intrinsics means that by lifting more non-relational code into the parts of the program where such things are visible to the optimizer, more code can be executed in C++ instead of in C\#.
So even though the physical plan has not changed, the resulting program might be might more efficient since it avoids the native to managed transition.


\section{Evaluation}
We analyze over 3,000,000 SCOPE jobs over a period of several days that run on five data centers at Microsoft. 
In summary, our experiments illustrate the following:

\begin{itemize}
\item \emph{What is the proportion of time spent in native vs. non-native job vertices?}
Between \nonNativeTimeL{} and \nonNativeTimeU \% of data center time is spent in job vertices that run managed code.

\item \emph{What proportion of time can be optimized having the current list of methods with C++ implementation?} 
With the current list of intrinsicable methods we can optimize up to \optimizableU{} \% of data center time.

\item \emph{What proportion of time can be optimized by extending the list of methods with C++ implementation? 
Which methods should be the most important for C++ implementation?}
By increasing the list of intrinsics and optimizing all inlineable methods we can optimize up to \potentiallyOptimizableU{} \% of data center time. 
Furthermore, we conclude that \emph{String} methods are the most important .NET framework methods amenable for C++ implementations.


\end{itemize}

\subsection{Experimental Setup}
To understand performance bottlenecks in SCOPE jobs we analyze over 3,000,000 jobs across 5 data centers.
Table~\ref{tb:projects} lists for every data center, the number of analyzed jobs along with their CPU time measured in hours.
We observe that number of jobs and CPU time significantly vary between data centers. 
For example, data center \emph{cosmos15} run the highest proportion of jobs we analyze, while \emph{cosmos9} run the most expensive jobs. 
This is expected because different data centers are usually tailored for different types of jobs. 


\begin{table}[ht]
\centering
\begin{tabular}{lrr}

  Data center & Number of jobs & CPU time (in hours) \\
 \midrule
cosmos8 & 375,974 & 28,559,063 \\
cosmos9 & 171,203 & 40,714,052 \\
cosmos11 & 851,222 & 23,312,271\\
cosmos14 & 474,911 & 21,299,039\\
cosmos15 & 1,200,026 & 31,324,407 \\
\midrule
Total: & 3,073,336 & 145,208,834\\
\midrule

\end{tabular}
 \label{tb:projects}
 \caption{Analyzed jobs and their CPU time}
\end{table}




\subsection{Native vs. Non-Native Time}
Before optimizing non-relational logic in SCOPE jobs, it is important to understand how much time is spent in job vertices that run non-relational code. 
The analysis of C++ code returns for every job vertex whether it runs as native or managed code. 
Combined with data from \emph{Runtime Statistics} we also find how much time is taken by every job vertex.

Figure~\ref{fig:nativeVsNonNative} illustrates our results.
It shows for every data center how much time is spent in native and non-native vertices. 
Grey area denotes few exceptional vertices for which the analysis can not certainly distinguish the code that runs inside a vertex. 
For example, sometimes the analysis does not detect any source of managed code in a vertex that supposedly runs C\# code. 
The time of such vertices we assign to gray area.

We conclude from Figure~\ref{fig:nativeVsNonNative} that time spent in non-native code represents a large fraction of data center time.
For example, in cosmos9, the time spent in native vertices contributes to only 21.30\% of data center time. 
These results illustrate the potential of optimizing the non-relational code and it influence on data center time.

\begin{figure*}[ht]
\includegraphics[width=2\columnwidth]{graphs/proportions.pdf}
\caption{Time spent in native vs. non-native vertices}
\label{fig:nativeVsNonNative}
\end{figure*}

\subsection{Optimizable Job Vertices}


We say a job vertex is \emph{optimizable} if it has as the only source of managed code inlineable methods that have only calls to intrinsics. 
Optimizable vertices are important because we can quantify how much data center time we can optimize considering the current list of intrinsics. 
Moreover, by inlining method calls, we expect an entire vertex to run as native code, which should significantly improve the vertex execution time.

Figure~\ref{fig:optimizable} shows the proportion of CPU time of \emph{optimizable} vertices relative to data center time and to time spent in vertices with non-native code. 
We observe that with the current list of intrinsics we can optimize a relatively small proportion of data center time. 
For example, in cosmos9 that runs the most expensive jobs, we can optimize at most 0.01\% of data center time. 
The situation is slightly better in cosmos14 and cosmos15, but in these data centers, the proportion of non-native time is relatively lower compared to cosmos9.

The crucial observation is that given results illustrate only the time in data centers that can be affected by inlining method calls in optimizable vertices. 
To measure the actual performance improvement it is necessary to rerun every optimized job. 
Further details on performance improvements for several jobs we optimize are given in Section~\ref{sec:caseStudy}.

\begin{figure}[ht]
\includegraphics[scale=0.8]{graphs/optimizable}
\caption{Optimizable job vertices}
\label{fig:optimizable}
\end{figure}

\subsection{Potential for C++ Translation}
To motivate the importance of providing the C++ implementation for more framework methods, we measure how much time is spent in the following type of vertices:
\begin{itemize}
\item Vertices with .NET framework calls as the only source of managed code
\item Vertices with .NET framework calls or calls to inlineable methods as the only source of managed code
\end{itemize}
We call these vertices \emph{potentially optimizable}, because they can run as native by increasing the list of intrinsics.

Figure~\ref{fig:potentially} shows the proportion of time spent in potentially optimizable vertices relative to data center time. 
We measure the proportions by assuming that all .NET framework methods have C++ implementation.
Results illustrate that we can optimize between \potentiallyOptimizableL{} and \potentiallyOptimizableU{} \% of data center time by just increasing the list of intrinsics. 
Even though \potentiallyOptimizableL{} \% of time spent in cosmos9 looks relatively low, it counts for almost 407,140 CPU hours for a period of several days. 
Knowing this type of impact motivates the future work on enabling more C++ translation of framework methods.

\begin{figure}[ht]
\includegraphics[scale=0.8]{graphs/potentiallyOptimizable}

\caption{Potentially optimizable job vertices}
\label{fig:potentially}
\end{figure}

\subsubsection{Most Relevant .NET Framework Methods}
Assuming that all .NET framework methods have C++ implementation is unrealistic. 
To provide more insights on the framework methods that actually matter, we perform two types of study:
the study of the most relevant methods considering the execution time of a vertex and the study of the most important method types. 

For the first study, we take all .NET framework methods called in potentially optimizable vertices and rank them based on the vertex execution time. 
Table~\ref{tb:rankedMethods} shows for every data center ten most important framework methods. 
The last row further illustrates how much data center time can be optimized if all methods in the list become intrinsics.
We further highlight methods that appear to be relevant across many data centers.
For example, \emph{System.String.ToLower} and \emph{System.String.Concat} are among the most relevant methods across all data centers.
Furthermore, if the native implementation is provided for the first ten methods in cosmos14, the data center it would be enough to optimize more than 5\% of data center time.

\begin{table*}[ht]
\small
 \begin{tabular}{@{}llllp{3.5cm}@{}}

  Cosmos8 & Cosmos9 & Cosmos11 & Cosmos14 & Cosmos15 \\
 \midrule
Convert.ToInt64 & \textbf{String.Equals} & String.Replace & \textbf{DateTime.ToString} & \textbf{String.ToLower} \\
Int32.Parse & \textbf{String.ToLower} & \textbf{String.ToLower} & String.IndexOf & String.LastIndexOf \\
\textbf{String.ToLower} & Int32.Parse & String.ToUpper & DateTime.ToLocalTime & \textbf{DateTime.ToString}\\
\textbf{String.Concat} & String.Replace & \textbf{String.Concat} & \textbf{String.ToLower} & \textbf{String.Concat}\\
String.Replace & Convert.ToDateTime & String.Trim & String.ToUpper & Convert.ToUInt64 \\
Double.Parse & Regex.isMatch & Math.Max & Regex.IsMatch & Enumerable.SelectMany \\
Math.Round& DateTime.ToUniversalTime & \textbf{String.Equals} & \textbf{String.Equals} & Enumerable.Distinct \\
Char.NewArr & \textbf{String.Concat} & TimeSpan.Days & \textbf{String.Concat} & String.Format \\
String.ToUpper & TimeSpan.Days & \textbf{DateTime.ToString} & String.Trim & \textbf{Syst.String.Equals}\\
String.Upper & DateTime.Subtract & String.ToCharArray & String.Split & String.IndexOf \\

\midrule
1.27\%\footnote{proportion relative to data center time} & 0.63\% & 1.61\% & 5.15\% & 1.8\%\\
\midrule

\end{tabular}
 
\caption{Most relevant .NET Framework methods per data center. All methods are within the {\tt System} namespace. Methods in bold are those that appear in the top 10 in at least 3 of the five data centers.
\label{tb:rankedMethods}}
\end{table*}


\begin{figure}[ht]
\includegraphics{graphs/methodTypes}
\caption{Relevance of .NET framework method types (cosmos11)}
\label{fig:methodTypes}
\end{figure}

Figure~\ref{fig:methodTypes} illustrates the most important method types for data center cosmos11. The results are comparable for other data centers. \emph{String} methods dominate and they count as the only source of non-native code in  1.49\% of the time spent in potentially optimizable vertices. Other method types are significantly less relevant, but when combined they influence 1.85\% of the data center time. These studies show the potential for improving data center performance by providing more intrinsics and thus enabling more C++ translation.







\section{Case Studies}
\label{sec:caseStudy}
In order to quantify the effects of optimizing the scripts, we performed a case study.
We say that a C\# method is {\em intrinsicable} if it is a .NET Framework method for which the SCOPE compiler has a semantically equivalent C++ function.
The jobs were chosen based on a static analysis that found {\em optimizable vertices}.
An optimizable vertex is one that is implemented in C\#, but the C\# code calls only intrinsicable methods or user-defined functions, {\em UDFs}, where the UDF, in turn, calls only intrinsicable methods, and does not call any other UDFs, i.e., our inlining depth is one.
We then manually looked at the top jobs from  a ranked list (by CPU time) of jobs containing an optimizable vertex.

Because the input data for each job is not available, we needed to contact the job owners and ask them to re-run the job with a manually-inlined version of their script.
We were able to have \casestudyjobs{} jobs re-run by their owners.
We roughly categorize the jobs by their total CPU time: short, medium, and long.

\subsection{Optimizations With Effects On Job Algebra}
As explained in Section~\ref{sec:intro}, the optimizer may choose to modify the job algebra given the new information available to it.
For example, predicates might be pushed deeper into the DAG which can result in dramatic data reduction.
However, none of the case studies ended up causing this kind of optimization.


\subsection{Optimizations Without Effects On Job Algebra}
Even if the physical plan does not change, the resulting program might be might more efficient if it avoids the native to managed transition.
For SCOPE, the set of intrinsics means that by lifting more non-relational code into the parts of the program where such things are visible to the optimizer, more code can be executed in C++ instead of in C\#.

\begin{figure*}[ht]
\begin{tabular}{c|c|c|c|c|c} 
\toprule
  {\em Job Name} & {\em C++ translation}&{\em Job Cost} & \multicolumn{2}{c}{\em CPU time} &  {\em Throughput } \\
  \cmidrule{4-5}  
  & & & {\em Vertex Change} & {\em Job Change} &  \\
  \midrule

A & yes & medium & 59.63\%  & 23.00\% & 30\%\\
B &yes & medium & no change & no change & no change\\
C & yes & low    & 41.98\%  & 25.00\% & 138\% \\
D & no & - & - & - & -\\
E & yes & high   & 7.22\%   & 4.79\% & \todo{} \\
F & yes & low & no change & no change & 215\%\\

%For Jobs A and B the numbers are averages over the a one week period.
\end{tabular}
\caption{Summary of case studies. The reported changes are percent improvement in CPU time and throughput. \label{fig:caseStudySummary}}
\end{figure*}

In total, we looked at \casestudyjobs{} re-run jobs, summarized in Figure~\ref{fig:caseStudySummary}. 
For one job (D), the optimization did not trigger C++ translation of an inlined operator because the operator called to a non-intrinsicable method that we mistakenly thought was an intrinsic.
After fixing this problem, we used the correct set of intrinsics to obtain data presented in Section~\ref{sec:evaluation}.
\begin{figure}[ht]
\includegraphics{graphs/normalizedTimesA}
\caption{Case Study A \label{fig:CaseStudyA}}
\end{figure}

\begin{figure}[ht]
\includegraphics{graphs/throughtputF}
\caption{Case Study F \label{fig:CaseStudyF}}
\end{figure}


For jobs A and B, we were able to perfom the historical study over a period of 18 days. 
Both jobs are medium-expensive jobs, run daily and contain exactly one optimizable vertex due to a UDF. In both cases, inlining that UDF resulted in the entire vertex being executed in C++.
Figure~\ref{fig:CaseStudyA} shows the CPU time for an optimized vertex in Job A over an 18 day period, the last 5 of which were with the inlined UDF.
The values are normalized by the average of the unoptimized execution times; the optimized version saves approximately 60\% of the execution time.
However, the normalized vertex CPU time in Job B does not show any consistent improvement for the last five jobs.
Closer analysis of the vertex shows that the operator which had been in C\# accounted for a very tiny percentage of the execution time for the vertex.
This is consistent with our results for Job A, where the operator had essentially been 100\% of the execution time of the vertex.


We also optimized Job F, a very low cost job.
It only runs a few times a month, so we were able to obtain timing information for only a few executions.
The vertex containing the optimized operator accounted for over 99\% of the overall CPU time for the entire job.
We found the CPU time to be highly variable; perhaps this is because the job runs so quickly so it is more sensitive to the batch environment in which it runs.
However,  we found the throughput measurements to be consistent: the optimized version provided twice the throughput for the entire job (again, compared to the average of the unoptimized version).

Finally, for jobs C and E we were not able to perform the same kind of historical study: instead we have just one execution of the optimized scripts. 
For this execution we found improvements in both vertex and job CPU times.


% Job A  is guid 74a1edb8
% Job B is guid 9b432578
% Job C is guid 01e72a39 and vertex SV4_Extract
% Job D is guid 0414cd51
% Job E is guid f2bd0961 and SV1_Extract_Combine_Split
% Job F is guid ac127d4f

\section{Threats to Validity}

\paragraph{Underapproximation of performance impact}
The amount of time that can be optimized by either increasing the list of intrinsics or method inlining is an underapproximation of the total optimizable time.
We do not consider effects of optimizations on another compiler optimizations.
For example, if a method is inlined on a column used to partition data, the inlining would not only trigger more C++ generation, it would also enable filter promotion~\cite{}.
Understanding the impact of inlining on another compiler optimizations is left for future work.

\paragraph{Assumptions for static analysis}
Our static analysis detects sources of C\# code based on several assumptions.
For example, we use naming conventions when pruning generated methods in C\# implementation of managed operators.
A user can potentially call some of these methods in the script, meaning that we would skip a valuable source of user-written C\# code.
However, in practice, such methods are not used in the context of big-data jobs and our manual exploration of many SCOPE scripts illustrates that our assumptions hold.

\paragraph{Challenges for implementing more intrinsics}
We discuss the relevance of providing C++ implementation for more .NET Framework methods.
However, providing C++ translation for some of these methods poses several challenges.
For example, memory management in C\# is very different because it has a garbage collector, while C++ does not.
Another challenge is related to different string encodings in C\# and C++ runtimes, and for some corner cases, there is no clear one-to-one mapping.
However, increasing the list of intrinsics would certainly bring significant performance benefits in SCOPE jobs, and there is a clear motivation for future work to address this problem.  

\section{Related Work}

\subsection{MapReduce Optimizations}
Since MapReduce~\cite{Dean:2008} is the de facto programming model for big data analytics~\cite{Gates:2009}, the performance of MapReduce is crucial to ensure optimal resource utilization and efficiency.
Various approaches have been proposed for the optimization of MapReduce jobs.
They include techniques to improve the scheduling of task execution~\cite{Isard:2009,Zaharia:2008}, to efficiently perform joins and indexing~\cite{Dittrich:2010,Floratou:2011}, an analysis for automatic work-sharing across multiple jobs~\cite{Nykiel:2010} and an extension to MapReduce model for efficiently merging the data computed by map and reduce modules~\cite{Yang:2007} . Although some of the standard database optimizations, such as filter pushdown are implemented in Pig~\cite{Olston:2008}, the recent work by Jahani et al.~\cite{Jahani:2011} suggests that many traditional query optimizations are not applicable for MapReduce because of the user-written operators. It further proposes several optimizations of \emph{map()} functions by targeting data-centric programming idioms~\cite{Jahani:2011}. Our work shows that time spent in non-relational code takes a large fraction of data center time and cross-language optimizations demonstrate a potential to significantly improve the performance of MapReduce jobs.

\subsection{Profiling MapReduce Jobs}
Many MapReduce systems, such as Hadoop~\cite{hadoop_stream}, provide facilities for monitoring cluster performance.
However, the collected metrics represent cluster-level information from which the regular user does not benefit.
Enabling users to optimize their jobs by tuning job parameters, Herodotou et al.~\cite{Herodotou} propose a dynamic binary instrumentation of the MapReduce framework to capture dataflows and costs during job execution at the task level or the phase level.
In contrast to the dynamic analysis, our approach to profiling big data jobs is purely static and based on the analysis of job artifacts. Doing this allows us to analyze a large number of jobs without introducing any additional overhead. 

\subsection{Query Optimizations}
There has been extensive work in optimizing relational queries since the early '70s~\cite{Chaudhuri:1998}. A large class of optimizations include exploiting commutativity among operators~\cite{Chaudhuri:1994,Yan:1995},reducing multi-block queries to single block~\cite{Kim:1982,Muralikrishna:1992}, using semijoin techniques to optimize multi-block queries~\cite{Mumick:1994,Seshadri:1996}, query materialization~\cite{Chaudhuri:1995,Phan:2008}, and query indexing~\cite{SELLIS1988175,Bertino:1989,Fang:2008}. Several approaches address optimizations of stored procedures (also called user-defined predicates) in relational systems~\cite{Hellerstein:1993,Chimenti:1989}. While these aforementioned techniques are powerful in optimizing relational queries, traditional query optimizers treat non-relational code as a black-box.



\section{Conclusions}

Big data systems are not as performant as they could be.
While this is true of all systems, the scale that such systems operate at means that any performance gap can translate into a large amount of money.
At the same time, large distributed systems are not as easy to measure and instrument as single-box systems.
Thus, we believe it crucial for new tools and processes to be developed for monitoring their performance.

We also believe having an expressive general-purpose language like C\# or Java integrated into a big data query language is a good thing: programmers should be able to re-use existing components in languages that they are already comfortable with.
However, such multi-language paradigms break the barriers that current program analysis and optimization tools are based on.

We have shown some techniques for measuring data center peformance and the utility of even small, simple optimizations.
Clearly, we have just scratched the surface of two very promising research areas.

\bibliographystyle{ACM-Reference-Format}
\bibliography{references} 

\end{document}
